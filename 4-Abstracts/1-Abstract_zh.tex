%----------------------------------------------------------------------
% 中文摘要
%----------------------------------------------------------------------

% 把中文摘要寫在裡面
\begin{zhAbstract}

    肺癌長年為台灣癌症死亡率第一名,其高度異質性與預後不確定性使得生存風險的預測極具挑戰性。數位病理影像作為腫瘤組織的高度表現載體,提供非侵入性的潛在預後資訊來源,深具臨床應用價值。近年來 Foundation Model 在醫療影像分析中展現強大潛力,然而在臨床生存分析等實務應用中,仍需針對任務特性進行適當調整與強化。

本研究旨在強化 Foundation Model 在肺癌病理影像生存風險預測上的表現,透過多尺度影像融合(multi-scale)、多模態特徵整合(multi-modal)以及低秩參數調整(Low-Rank Adaptation, LoRA)進行優化。首先,我們針對切片影像設計多尺度輸入機制,結合低放大倍率與高放大倍率的影像視角,以交叉注意力的方式進行特徵融合,協助模型從不同層次理解腫瘤結構與其周圍微環境。其次,為了進一步提升模型對腫瘤生物學異質性的感知能力,我們引入細胞層級的形態資訊,與病理影像特徵進行融合,使模型能同時捕捉巨觀與微觀層次的預後相關線索。

在模型架構上,採用大型 Foundation Model 作為 encoder,並結合 gated-attention MIL 機制,以弱監督方式完成病人層級的生存風險預測。MIL 架構可自動聚焦在具有預後意義的局部區域,搭配 Foundation Model 的深層特徵,強化對關鍵腫瘤區域的學習。此外,我們導入 LoRA 技術對 Foundation Model 進行低秩微調,使模型能以極少參數更新達成任務適應,兼顧訓練效率與泛化能力,特別適用於醫療資料樣本數有限的情境。

實驗資料使用 TCGA-LUAD 病理影像與臨床資訊,共計約 300 位病患,模型以 Cox Proportional Hazards 作為損失函數訓練,最終在測試集上可達 C-index 0.66,顯著優於未經調整的 Foundation Model。整體而言,本研究展示如何透過多尺度與多模態特徵融合、MIL 機制與 LoRA 微調技術,有效提升 Foundation Model 在肺癌生存分析任務中的適應性與預測效能,為未來智慧病理應用提供新的研究方向與模型範式。
    % 這個Command會自動幫你把Config裡面設定的東東填進來
    \zhAbsKeywords
\end{zhAbstract}
